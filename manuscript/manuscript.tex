% interactcadsample.tex
% v1.03 - April 2017

\documentclass[]{interact}

\usepackage{epstopdf}% To incorporate .eps illustrations using PDFLaTeX, etc.
\usepackage{subfigure}% Support for small, `sub' figures and tables
%\usepackage[nolists,tablesfirst]{endfloat}% To `separate' figures and tables from text if required

\usepackage{natbib}% Citation support using natbib.sty
\bibpunct[, ]{(}{)}{;}{a}{}{,}% Citation support using natbib.sty
\renewcommand\bibfont{\fontsize{10}{12}\selectfont}% Bibliography support using natbib.sty

\theoremstyle{plain}% Theorem-like structures provided by amsthm.sty
\newtheorem{theorem}{Theorem}[section]
\newtheorem{lemma}[theorem]{Lemma}
\newtheorem{corollary}[theorem]{Corollary}
\newtheorem{proposition}[theorem]{Proposition}

\theoremstyle{definition}
\newtheorem{definition}[theorem]{Definition}
\newtheorem{example}[theorem]{Example}

\theoremstyle{remark}
\newtheorem{remark}{Remark}
\newtheorem{notation}{Notation}

% see https://stackoverflow.com/a/47122900

\usepackage{hyperref} \usepackage[utf8]{inputenc} \def\tightlist{}

\begin{document}

\articletype{WIP DRAFT}

\title{Towards Explainability of Machine Learning Models in Insurance Pricing}


\author{\name{Author One$^{a}$, Author Two$^{b}$}
\affil{$^{a}$Foo; $^{b}$Bar}
}

\thanks{CONTACT Author One. Email: \href{mailto:foo@bar.foo}{\nolinkurl{foo@bar.foo}}, Author Two. Email: \href{mailto:bar@foo.bar}{\nolinkurl{bar@foo.bar}}}

\maketitle

\begin{abstract}
Abstract tbd
\end{abstract}

\begin{keywords}
actuarial science; general insurance
\end{keywords}

\section{Introduction}\label{introduction}

Risk classification for property \& casualty (P\&C) insurance rating has
traditionally been done with one-way, or univariate, analysis
techniques. In recent years, many insurers have moved towards using
generalized linear models (GLM), a multivariate predictive modeling
technique, which addresses many shortcomings of univariate approaches,
and is currently considered the gold standard in insurance risk
classification. At the same time, machine learning (ML) techniques such
as deep neural networks have gained popularity in many industries due to
their superior predictive performance over linear models
\citep{lecunDeepLearning2015}. In fact, there is a fast growing body of
literature on applying ML to P\&C reserving
\citep{kuoDeepTriangleDeep2018, wuthrichMachineLearning2018, gabrielliNeuralNetwork2019a, gabrielliNeuralNetwork2019}.
However, these ML techniques, often considered to be completely ``black
box'', have been less successful in gaining adoption in pricing, which
is a regulated discipline and requires a certain amount of transparency
in models.

If insurers can gain more insight into how ML models behave in risk
classification contexts, it would increase their ability to reassure
regulators and the public that accepted ratemaking principles are met.
Being able to charge more accurate premiums would, in turn, make the
risk transfer system more efficient and contribute to the betterment of
society. In this paper, we aim to take a step towards liberating
actuaries from the confines of linear models in pricing projects, by
proposing a framework for explaining ML models for ratemaking that
regulators, practitioners, and researchers in actuarial science can
build upon.

The rest of this paper is organized as follows: Section \ref{ratemaking}
provides an overview of P\&C ratemaking, Section \ref{interpretability}
discusses model interpretability in the context of ratemaking and
proposes specific tasks for model explanation, Section \ref{application}
describes current model interpretation techniques and applies them to
the tasks defined in the previous section, and Section \ref{conclusion}
concludes.

\section{Property and Casualty Ratemaking}\label{ratemaking}

\subsection{History of Ratemaking}\label{history-of-ratemaking}

Early classification ratemaking procedures were typically univariate in
nature. For example, \citep{lange_1966} notes that (at that time) most
major lines of insurance used univariate methods based around the same
principle: distributing an overall indication to territorial
relativities or classification relativities based on the extent to which
they deviated from the average experience.

\citep{bailey_simon_1960} introduced minimum bias methods, which were
expanded throughout the 60s, 70s, and 80s. As computing power developed,
minimum bias began to give away to generalized linear models, with
papers such as \citep{brown_1988} and \citep{mildenhall_1999} bridging
the gap between the methods.

Arguably, generalized linear models predate minimum bias procedures by a
significant margin. The term ``Generalized Linear Model'' was coined by
\citep{nelder_wedderburn_1972}, but generalizations of least squares
linear regression date back at least to the 1930s. Like minimum bias
methods, GLMs did not become mainstream in actuarial science for some
time. For example, the syllabus of basic education does not seem to
include any mention of GLMs prior to \citep{brown_1988} in the 1990
syllabus for basic education. From there, GLMs seem to have received
only passing mention until 2006 with the introduction of
\citep{anderson_2005} to the syllabus.

\subsection{Machine Learning in
Ratemaking}\label{machine-learning-in-ratemaking}

Paralelling the development of generalized linear models was the
development of machine learning algorithms throughout the middle part of
the 20th century. Detailed histories of machine learning may be found in
sources such as \citep{Nilsson_2009} and \citep{wang_raj_2017}.
Consistent with GLMs, machine learning was relatively unpopular in
actuarial science until the last ten years as computing power has become
cheaper and more easily available and as machine learning software
packages have obviated the need for developing analyses from scratch
each time an analysis is performed. Due to the breadth of machine
learning as a field, it is difficult to identify the first time it
entered the CAS syllabus; however, cluster analysis (in the form of
k-means) seems to have been first included in 2011 with
\citep{robertson_2009}. More recently, the MAS-I and MAS-II exams
introduced in 2018 have included machine learning explicitly.

Within the area of ratemaking, machine learning is still in its infancy.
A significant portion of machine learning applications to ratemaking has
been in the context of automobile telematics, such as \citep{gao_2018},
\citep{gao_2018_2}, \citep{gao_2019}, \citep{roel_2018}, or
\citep{wuthrich_2017}. Presumably this focus has been a result of the
high-dimensionality and complexity of telematics data, making it a field
in which the unique abilities of machine learning techniques give a
clear advantage over traditional approaches.

Outside of telematics, \citep{yang_2018} uses a gradient tree-boosting
approach to capture non-linearities that would be a challenge for GLMs.
\citep{henckaerts_2018} makes use of ``generalized additive models'' to
improve predictions of GLMs. Many researchers, in an apparent effort to
demonstrate the range of possibilities and advantages of machine
learning, have approached the topic by comparing many different machine
learning algorithms within a single study, such as in
\citep{dugas_2003}, \citep{noll_2018}, \citep{spedicato_2018}. These
studies make use of such varied techniques as regression trees, boosting
machines, support vector machines, and neural networks.

\subsection{Ratemaking Process}\label{ratemaking-process}

Regardless of the method employed for determining this risk of various
classifications, the actual process of setting rate relativities
typically involves some variation of the following steps:

\begin{enumerate}
\def\labelenumi{\arabic{enumi}.}
\tightlist
\item
  Obtain relevant policy-level data
\item
  Prepare data for analysis
\item
  Perform analysis on the data, employing desired method or methods to
  estimate needed rate relativites
\item
  Select final rate relativities based on rate indications
\item
  Present rates to the reglator, including explanation of the steps
  followed to derive the rates
\item
  Answer questions from regulators regarding the method employed
\end{enumerate}

The focus of this paper is on steps 5 and 6. In particular, rate
regulators are concerned with whether rates are inadequate, excessive,
or unfairly discriminatory. In many states, rate filings that exceed
certain thresholds for magnitude of rate changes or filings that make
use of new or sophisticated predictive models may be subject to intense
regulatory scrutiny. In these cases, it is necessary to be able to
explain the results of the modeling process in a way that is
understandable without sacrificing statistical rigor.

It should be noted that communicating results is not simply a method of
passing regulatory muster. Generating interpretable modeling output is
an important - even essential - facet of model checking. Therefore, the
techniques discussed in this paper may be viewed from the lens of
providing useful information to regulators, but they should also be
considered as part of a thorough vetting of any rating model.

\section{Interpretability in the Ratemaking
Context}\label{interpretability}

(literature review, settle on a definition to work with, e.g.
\citet{doshi-velezRigorousScience2017}, identify maybe 3 questions about
a model to answer, tie to principles on p/c ratemaking)

Within the actuarial profession, Actuarial Standard of Practice 41
(``Actuarial Communications'') notes that ``\ldots{}another actuary
qualified in the same practice area {[}should be able to{]} make an
objective appraisal of the reasonableness of the actuary's work as
presented in the actuarial report.'' \citep{asop_41} Underlying this
requirement is an assumption that the hypothetical other actuary
qualified in the same practice area is adequately familiar with the
relevant techniques employed. Although the syllabus of basic education
is constantly changing, there has at times been an assumption that all
techniques and assumptions that have ever been a part of the syllabus of
basic education needn't be explained from first principles in general
actuarial communications, and that an actuary practicing in the same
field should be able to make an objective appraisal of the results from
the methods found in the syllabus. {[}This can be supported by reference
to the edits to ASOP 38 over time - originally it was designed to be
about all models, but they revised it to be about only models that
incorporate specialized knowledge outside of the actuary's area of
expertise\ldots{} do we need this citation, though?{]} This is notable
because, beginning with the introduction of the MAS-I and MAS-II
examination in July of 2018, several machine learning models were
formally included in the syllabus of basic education. These exams cover
a wide range of topics, such as splines, clustering algorithms, decision
trees, boosting, and principle components analysis.
\citep{cas_syllabus_2018}

Nevertheless, machine learning poses something of a special challenge
for ASOP 41 for several reasons. Machine learning models can be very ad
hoc compared to traditional statistical models. Because many machine
learning models are deterministic, they may not admit of standard
metrics for model comparison (e.g., it's not straightforward to
calculate an AIC over a neural network). In addition, machine learning
methods are often combined into ensembles that may not be easily
separated and that may, as a collection, cease to resemble a single
standard version of a model. Complicating matters still further, machine
learning models can be ``black boxes'' insofar as the final form of
response curve cannot be easily predicted and may depend heavily on the
available data (which may not, in turn, be available to the reviewer).

This last item raises a final interesting issue. Generalized linear
models and their ilk are often fitted using one of a handful of standard
and well-understood approaches (e.g., maximum likelihood estimation).
However, this is not possible in general with machine learning models,
as machine learning algorithms often use loss surfaces that are very
complex such that it may not be feasible to calculate the global minimum
of the surface. Certainly, closed form representations of the loss
surfaces are not generally available. For this reason, the training
phase of a machine learning model is, in many ways, just as important to
one's understanding as the model form and the data on which the model is
fitted. Because the final model result is inseparable from these three
components (training method, model form, and data), it is not generally
adequate to just know the method employed to make an objective appraisal
of the reasonableness of the model. More information is necessary.

Of course, these comments only apply to the actuarial profession.
Outside of the actuarial profession, communication of results may be
more challenging. A 2017 survey conducted by the Casualty Actuarial and
Statistical Task Force of the National Association of Insurance
Commissioners found that the plurality of responding regulators
identified ``Filing complexity and/or a lack of resources or expertise''
as a key challenge that impedes their ability to review GLMs or other
predictive models. \citep{naic_summer_2017} Given that machine learning
algorithms are generally regarded as more complex than GLMs, this
implies that the challenge of communicating machine learning model
results is significant.

In response to the same survey, 33 state regulators noted that it would
be helpful or very helpful for the NAIC to develop information and tools
to assist in reviewing rate filings based on GLMs, and 34 noted that it
would be helpful to develop similar items to assist in reviewing ``Other
Advanced Modeling Techniques.'' One outgrowth of this need was the
development of a white paper on best practices for regulatory review of
predictive models. The white paper focuses on review of GLMs,
particularly with respect to private passenger automobile and
homeowners' insurance. Some of the guidance offered in this regard is
therefore not strictly applicable to the review of machine learning
models. For example, as previously noted, p-values are not a concept
that translates well to deterministic machine learning algorithms.
However, among the guidance applicable to machine learning algorithms
are the following (paraphrasing):

\begin{itemize}
\tightlist
\item
  understand the relationship between the inputs and the expected loss
  or expense differences in risk,
\item
  determine that these input characteristics are not unfairly
  discriminatory, and
\item
  determine the extent of premium disruption among policyholders and be
  able to explain the source of premium disruptions.
  \citep{naic_white_paper}
\end{itemize}

Note that this list is not exhaustive of the guidance offered by any
means, but that these three items are those that may present greater
challenge for machine learning algorithms compared to GLMs.

Depending on the model, machine learning algorithms can be non-linear in
the inputs. In areas where data are sparse, the model could generate
unnatural response curves that could go undetected by a very high-level
view of the model results. This leads to a situation where lift charts
or tests on holdout data may indicate that the model is performing
adequately, but it may be difficult to explain the reasons for a
particular premium indicaton or a change in premium when policy
characteristics change.

-For this, we recommend \_\_\_\_ (grid search?) -Because the model form
may be difficult to represent in the form of a single equation (and
harder still to evaluate), one method of evaluating the reasonableness
of model results is local approximations of the impact of different
rating variables (LIME?)

\section{Applying Model Interpretation Techniques}\label{application}

(some definitions, e.g.~global/model vs.~local/instance, categorize
questions accordingly)

\subsection{(answer each question)}\label{answer-each-question}

(for each question, propose a technique, mention alternative techniques,
pros/cons, implement, interpret results)

\section{Conclusion}\label{conclusion}

(conclude)

\bibliographystyle{tfcad}
\bibliography{explain-pricing.bib}


\input{"appendix.tex"}


\end{document}
